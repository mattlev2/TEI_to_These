%%%%Pr�ambule du fichier principal: les lignes suivantes, avant \begin{document} et le corps du texte:
%\documentclass[francais,twoside,a4paper]{book}
%R\UTF{00E9}soud le probl\UTF{00E8}me de registre (No room for...)
%\usepackage{etex}
%\reserveinserts{28}
%%%%%%%%%%r\UTF{00E9}soud le probleme de registre%%%%%%%
%Utiliser un fichier externe pour la bibliographie%
%\usepackage{subfiles}
%Utiliser un fichier externe pour la bibliographie%
%\input{chemin du pr�ambule}



\usepackage[T1]{fontenc} 
\usepackage{fbb}
\usepackage[latin1]{inputenc}
\usepackage[francais]{babel}
\usepackage{lipsum}
%Diff�rents types de guillemets selon la tradition du pays
\usepackage{csquotes}
%Diff�rents types de guillemets selon la tradition du pays
\usepackage{bookmark}
\usepackage{caption}
\widowpenalty=9999
\usepackage{url}
%contr�le de la mise en page g�n�rale
\usepackage{layout}
%contr�le de la mise en page g�n�rale

%couleurs
\usepackage[x11names, HTML,dvipsnames]{xcolor}
%couleurs

%Bonne hyphenation des urls
\makeatletter
\g@addto@macro{\UrlBreaks}{\UrlOrds}
\makeatother
%Bonne hyphenation des urls

\usepackage[top=2.5cm, bottom=1.5cm, left=2.25cm,right=3.25cm, heightrounded, marginparwidth=2.5cm, marginparsep=0.8cm, includefoot]{geometry}

%ESPACE ENTRE LES LIGNES
%\DisemulatePackage{setspace}%n\UTF{00E9}cessaire pour que \UTF{00E7}a marche avec memoir
\usepackage{setspace}
\onehalfspacing
%ESPACE ENTRE LES LIGNES

%%%%%%%%%%%%%%%%%%%%%%%%%%%STRUCTURE%%%%%%%%%%%%%%%%%%%%%%%%%%%%%%%%%%%%
%Je sais pas � quoi �a sert
\usepackage{etoolbox}
%Je sais pas � quoi �a sert


\usepackage{ifthen}

%%% HEADER FOOTER

%profondeur de la num�rotation structurelle
\setcounter{secnumdepth}{0}
%profondeur de la num�rotation structurelle

%ESPACEMENT DU TITRE
%\titlespacing*{\tableofcontents}{-90pt}{-70pt}{10pt}
%ESPACEMENT DU TITRE


%%%%%Red\UTF{00E9}finition des diff\UTF{00E9}rents titres de section%%%%%
%\titleformat{\chapter}[display]
  %  {\normalfont\huge\bfseries}{\chaptertitlename\ \thechapter}{10pt}{\huge}
%\titlespacing*{\chapter}{-20pt}{-50pt}{10pt}%% > haut de page. La deuxi\UTF{00E8}me fenetre de r\UTF{00E9}glage fait la hauteur avant le titre, la premi\UTF{00E8}re gauche/droite, la derni\UTF{00E8}re la hauteur apr\UTF{00E8}s le titre. 
%\titleformat{\section}
%  {\normalfont\LARGE\bfseries}{\thesection}{1em}{}
%\titlespacing*{\section}{-20pt}{40pt}{10pt}
%\interfootnotelinepenalty=10000
%\titleformat{\subsection}
%  {\normalfont\scshape\Large\bfseries}{\thesubsection}{1em}{}
%  \titlespacing*{\subsection}{10pt}{30pt}{10pt}
%\titleformat{\subsubsection}
%  {\sffamily\large\itshape%\filcenter
  %}{\thesubsubsection}{1em}{}
  %\titlespacing*{\subsubsection}{5pt}{20pt}{30pt}
%%%%%Red\UTF{00E9}finition des diff\UTF{00E9}rents titres de section%%%%%

%Grosseur d'indentation
\parindent=1cm
%Grosseur d'indentation

%%%%%%%%%%%%%%%%%%%%%%%%%%%STRUCTURE%%%%%%%%%%%%%%%%%%%%%%%%%%%%%%%%%%%%

 %%%MARGE DES CITATIONS (\BEGIN{NARROWTEXT}%%%
\makeatletter
\newcommand*\narrowtext[1][0.2\linewidth]{%UN EQUIVALENT DE {quotation} QUI PERMET DE MODIFIER LA TAILLE DES MARGES
    \begingroup\medbreak%TAILLE DU SAUT AVANT PARAGRAPHE
    \leftskip\ifx\@empy#1\@empty0.2\linewidth\else#1\fi\relax
    \@testopt\narrowtext@i{0.2\linewidth}}
\newcommand*\narrowtext@i[1][0.2\linewidth]{%
    \rightskip\ifx\@empy#1\@empty\leftskip\else#1\fi\relax\ignorespaces}
\def\endnarrowtext{\medbreak\endgroup}%TAILLE DU SAUT APRES PARAGRAPHE
\makeatother
  %%%MARGE DES CITATIONS (\BEGIN{NARROWTEXT}%%%

% Commencer les chapitres de l'�dition  et la biblio en page impaire (ajouter \cleartorightpage dans la template d'impression du titre de chaque chapitre)
%\makeatletter
%\def\cleartorightpage{\clearpage\if@twoside \ifodd\c@page\else
%\hbox{}\thispagestyle{empty}\newpage\fi\fi}
%\makeatother
%\makeatletter
%\def\cleartorightpage{\clearpage\if@twoside \ifodd\c@page\else
%\hbox{}\thispagestyle{biblio}\newpage\fi\fi}
%\makeatother
%\makeatletter
%\def\cleartorightpage{\clearpage\if@twoside \ifodd\c@page\else
%\hbox{}\thispagestyle{toc}\newpage\fi\fi}
%\makeatother
% Commencer les chapitres de l'�dition en page impaires

%%%%%%%%%%%%%%%%%%%%%%%%%%%NOTES DE BAS DE PAGE ET APPARAT%%%%%%%%%%%%%%%%%%%%%%%%%%

%%%%Notes de bas de page: appel non superscript, appel de note dans le text en exposant.
\makeatletter
\renewcommand\@makefntext[1]%
    {\noindent\makebox[0pt][r]{{\@thefnmark}\,. }#1}
\makeatother
%%%%Notes de bas de page: appel non superscript, appel de note dans le text en exposant.

%sets the footnote marker flush with,  but just inside the margin from, the text of the footnote; This option forces footnotes to the bottom of the page
\usepackage[bottom,flushmargin]{footmisc}
%sets the footnote marker flush with,  but just inside the margin from, the text of the footnote; This option forces footnotes to the bottom of the page

%%%%Indentation Parfaite Des Notes De Bas De Page
\makeatletter
\long\def\@makefntextFB#1{%
    \ifx\thefootnote\ftnISsymbol
        \@makefntextORI{#1}%
    \else
        \rule\z@\footnotesep
        \setbox\@tempboxa\hbox{\@thefnmark}%
            \ifdim\wd\@tempboxa>\z@
                \kern2em\llap{\@thefnmark.\kern0.5em}%
            \fi
        \hangindent2em\hangafter\@ne#1
    \fi}
\makeatother
%%%%Indentation Parfaite Des Notes De Bas De Page

\newcommand{\hsp}{\hspace{20pt}}
\newcommand{\HRule}{\rule{\linewidth}{0.5mm}}

%%%%%%%%%%%%%%%%%%%%%%%%%%%%%%%%NOTES DE BAS DE PAGE%%%%%%%%%%%%%%%%%%%%%%%%%%%%

%%%%%%%%%%%%%%%%%%%%%%%%%%%%%%%%%BIBLIO%%%%%%%%%%%%%%%%%%%%%%%%%%%%%%%%%%%%%
\usepackage{xpatch}
\usepackage{filecontents}
\usepackage[style=verbose-trad1,citestyle=authortitle, isbn=false, doi=false,backend=bibtex8,language=french,url=false]{biblatex}
%\renewcommand*{\mkibid}{\emph}
\DeclareFieldFormat[report]{title}{\textit{{#1}}}
\DeclareFieldFormat[article]{journaltitle}{{\textit{{#1}}}}
  \DeclareFieldFormat[book]{title}{{\textit{#1}}}
    \DeclareFieldFormat[misc]{title}{{\textit{#1}}}
    \DeclareFieldFormat[book]{booktitle}{{\textit{#1}}}
   \DeclareFieldFormat[inbook]{booktitle}{{\textit{#1}}}
  \DeclareFieldFormat[techreport]{title}{{\textit{#1}}}
\renewcommand{\mkbibnamelast}[1]{\textsc{#1}}
\DeclareLanguageMapping{francais}{francais-apa}

%change In: par \textit{in}
\DefineBibliographyStrings{french}{
  in = {\textit{in}},
}
\renewbibmacro{in:}{%
  \ifentrytype{article}{}{\printtext{\bibstring{in}\intitlepunct}}}
%change In: par \textit{in}

%ponctuation entre  \UTF{00E9}l\UTF{00E9}ments: virgule
    \renewcommand{\newunitpunct}[0]{, }
%ponctuation entre  \UTF{00E9}l\UTF{00E9}ments: virgule

\makeatletter
\AtEveryCitekey{%
  \ifboolexpr{ test {\iffieldequalstr{entrysubtype}{classical}}
               and not test {\iffieldundef{shorttitle}} }
    {\ifciteseen
       {\blx@ibidreset\clearname{labelname}}
       {\savefield{title}{\cbxtitle}\restorefield{labeltitle}{\cbxtitle}}}
    {}}
\makeatother


%%%%%%%%%%%%%%%%%%%%%%%%%%%%%%%%%BIBLIO%%%%%%%%%%%%%%%%%%%%%%%%%%%%%%%%%%%%%

\subfile{/Users/Matt/Desktop/Pre-these/Projet_de_these/Sortie_LaTeX/Bibliographie/Biblio_projet_de_these.tex}
\addbibresource{\jobname.bib}
\title{Bibliographie}
%\usepackage[urlcolor=green]{hyperref}%r\UTF{00E9}f\UTF{00E9}rences internes au document (toc, etc) \UTF{00E0} mettre en dernier

