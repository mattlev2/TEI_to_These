 %%!TEX TS-program = arara
% arara: pdflatex: { shell: yes }
% arara: bibtex
% arara: pdflatex: {shell: yes }  
%Arrara marche comme suit: Un % pour faire fonctionner une ligne, deux %% pour la commenter. 
\documentclass[francais,oneside,a4paper,11pt]{article}
%R\UTF{00E9}soud le probl\UTF{00E8}me de registre (No room for...)
\usepackage{etex}
\reserveinserts{28}
%%%%%%%%%%r\UTF{00E9}soud le probleme de registre%%%%%%%

%N�cessaire pour g�rer la bibliographie dans un fichier externe%%
\usepackage{subfiles}

%N�cessaire pour g�rer la bibliographie dans un fichier externe%%

\usepackage[T1]{fontenc} 
\usepackage{fbb}
\usepackage[latin1]{inputenc}
\usepackage[francais]{babel}
\usepackage{lipsum}
%Diff�rents types de guillemets selon la tradition du pays
\usepackage{csquotes}
%Diff�rents types de guillemets selon la tradition du pays
\usepackage{bookmark}
\usepackage{caption}
\widowpenalty=9999

%contr�le de la mise en page g�n�rale
\usepackage{layout}
%contr�le de la mise en page g�n�rale

%couleurs
\usepackage[x11names, HTML,dvipsnames]{xcolor}
%couleurs

%Bonne hyphenation des urls
\makeatletter
\g@addto@macro{\UrlBreaks}{\UrlOrds}
\makeatother
%Bonne hyphenation des urls

\usepackage[top=2.5cm, bottom=1.5cm, left=2.25cm,right=2.25cm, heightrounded, marginparwidth=2.5cm, marginparsep=0.8cm, includefoot]{geometry}

%ESPACE ENTRE LES LIGNES
%\DisemulatePackage{setspace}%n\UTF{00E9}cessaire pour que \UTF{00E7}a marche avec memoir
\usepackage{setspace}
\onehalfspacing
%ESPACE ENTRE LES LIGNES

%%%%%%%%%%%%%%%%%%%%%%%%%%%STRUCTURE%%%%%%%%%%%%%%%%%%%%%%%%%%%%%%%%%%%%
%Je sais pas � quoi �a sert
\usepackage{etoolbox}
%Je sais pas � quoi �a sert


\usepackage{ifthen}

%%% HEADER FOOTER

%profondeur de la num�rotation structurelle
\setcounter{secnumdepth}{0}
%profondeur de la num�rotation structurelle

%ESPACEMENT DU TITRE
%\titlespacing*{\tableofcontents}{-90pt}{-70pt}{10pt}
%ESPACEMENT DU TITRE


%%%%%Red\UTF{00E9}finition des diff\UTF{00E9}rents titres de section%%%%%
%\titleformat{\chapter}[display]
  %  {\normalfont\huge\bfseries}{\chaptertitlename\ \thechapter}{10pt}{\huge}
%\titlespacing*{\chapter}{-20pt}{-50pt}{10pt}%% > haut de page. La deuxi\UTF{00E8}me fenetre de r\UTF{00E9}glage fait la hauteur avant le titre, la premi\UTF{00E8}re gauche/droite, la derni\UTF{00E8}re la hauteur apr\UTF{00E8}s le titre. 
%\titleformat{\section}
%  {\normalfont\LARGE\bfseries}{\thesection}{1em}{}
%\titlespacing*{\section}{-20pt}{40pt}{10pt}
%\interfootnotelinepenalty=10000
%\titleformat{\subsection}
%  {\normalfont\scshape\Large\bfseries}{\thesubsection}{1em}{}
%  \titlespacing*{\subsection}{10pt}{30pt}{10pt}
%\titleformat{\subsubsection}
%  {\sffamily\large\itshape%\filcenter
  %}{\thesubsubsection}{1em}{}
  %\titlespacing*{\subsubsection}{5pt}{20pt}{30pt}
%%%%%Red\UTF{00E9}finition des diff\UTF{00E9}rents titres de section%%%%%

%Grosseur d'indentation
\parindent=1cm
%Grosseur d'indentation

%%%%%%%%%%%%%%%%%%%%%%%%%%%STRUCTURE%%%%%%%%%%%%%%%%%%%%%%%%%%%%%%%%%%%%

 %%%MARGE DES CITATIONS (\BEGIN{NARROWTEXT}%%%
\makeatletter
\newcommand*\narrowtext[1][0.2\linewidth]{%UN EQUIVALENT DE {quotation} QUI PERMET DE MODIFIER LA TAILLE DES MARGES
    \begingroup\medbreak%TAILLE DU SAUT AVANT PARAGRAPHE
    \leftskip\ifx\@empy#1\@empty0.2\linewidth\else#1\fi\relax
    \@testopt\narrowtext@i{0.2\linewidth}}
\newcommand*\narrowtext@i[1][0.2\linewidth]{%
    \rightskip\ifx\@empy#1\@empty\leftskip\else#1\fi\relax\ignorespaces}
\def\endnarrowtext{\medbreak\endgroup}%TAILLE DU SAUT APRES PARAGRAPHE
\makeatother
  %%%MARGE DES CITATIONS (\BEGIN{NARROWTEXT}%%%

% Commencer les chapitres de l'�dition  et la biblio en page impaire (ajouter \cleartorightpage dans la template d'impression du titre de chaque chapitre)
%\makeatletter
%\def\cleartorightpage{\clearpage\if@twoside \ifodd\c@page\else
%\hbox{}\thispagestyle{empty}\newpage\fi\fi}
%\makeatother
%\makeatletter
%\def\cleartorightpage{\clearpage\if@twoside \ifodd\c@page\else
%\hbox{}\thispagestyle{biblio}\newpage\fi\fi}
%\makeatother
%\makeatletter
%\def\cleartorightpage{\clearpage\if@twoside \ifodd\c@page\else
%\hbox{}\thispagestyle{toc}\newpage\fi\fi}
%\makeatother
% Commencer les chapitres de l'�dition en page impaires

%%%%%%%%%%%%%%%%%%%%%%%%%%%NOTES DE BAS DE PAGE ET APPARAT%%%%%%%%%%%%%%%%%%%%%%%%%%

%%%%Notes de bas de page: appel non superscript, appel de note dans le text en exposant.
\makeatletter
\renewcommand\@makefntext[1]%
    {\noindent\makebox[0pt][r]{{\@thefnmark}\,. }#1}
\makeatother
%%%%Notes de bas de page: appel non superscript, appel de note dans le text en exposant.

%sets the footnote marker flush with,  but just inside the margin from, the text of the footnote; This option forces footnotes to the bottom of the page
\usepackage[bottom,flushmargin]{footmisc}
%sets the footnote marker flush with,  but just inside the margin from, the text of the footnote; This option forces footnotes to the bottom of the page

%%%%Indentation Parfaite Des Notes De Bas De Page
\makeatletter
\long\def\@makefntextFB#1{%
    \ifx\thefootnote\ftnISsymbol
        \@makefntextORI{#1}%
    \else
        \rule\z@\footnotesep
        \setbox\@tempboxa\hbox{\@thefnmark}%
            \ifdim\wd\@tempboxa>\z@
                \kern2em\llap{\@thefnmark.\kern0.5em}%
            \fi
        \hangindent2em\hangafter\@ne#1
    \fi}
\makeatother
%%%%Indentation Parfaite Des Notes De Bas De Page

\newcommand{\hsp}{\hspace{20pt}}
\newcommand{\HRule}{\rule{\linewidth}{0.5mm}}

%%%%%%%%%%%%%%%%%%%%%%%%%%%%%%%%NOTES DE BAS DE PAGE%%%%%%%%%%%%%%%%%%%%%%%%%%%%

%%%%%%%%%%%%%%%%%%%%%%%%%%%%%%%%%BIBLIO%%%%%%%%%%%%%%%%%%%%%%%%%%%%%%%%%%%%%
\usepackage{xpatch}
\usepackage{filecontents}
\usepackage[style=verbose-trad1,citestyle=authortitle-ibid, isbn=false, doi=false,backend=bibtex8,language=french,url=false]{biblatex}
%Pas de pagebreak avant la biblio
\defbibheading{secbib}[\bibname]{%
 % \section*{#1}%
 % \markboth{#1}{#1}
 }
  %Pas de pagebreak avant la biblio
%\renewcommand*{\mkibid}{\emph}
\DeclareFieldFormat[report]{title}{\textit{{#1}}}
\DeclareFieldFormat[article]{journaltitle}{{\textit{{#1}}}}
  \DeclareFieldFormat[book]{title}{{\textit{#1}}}
    \DeclareFieldFormat[misc]{title}{{\textit{#1}}}
    \DeclareFieldFormat[book]{booktitle}{{\textit{#1}}}
   \DeclareFieldFormat[inbook]{booktitle}{{\textit{#1}}}
  \DeclareFieldFormat[techreport]{title}{{\textit{#1}}}
\renewcommand{\mkbibnamelast}[1]{\textsc{#1}}
\DeclareLanguageMapping{francais}{francais-apa}

%change In: par \textit{in}
\DefineBibliographyStrings{french}{
  in = {\textit{in}},
}
\renewbibmacro{in:}{%
  \ifentrytype{article}{}{\printtext{\bibstring{in}\intitlepunct}}}
%change In: par \textit{in}

%ponctuation entre  \UTF{00E9}l\UTF{00E9}ments: virgule
    \renewcommand{\newunitpunct}[0]{, }
%ponctuation entre  \UTF{00E9}l\UTF{00E9}ments: virgule

\makeatletter
\AtEveryCitekey{%
  \ifboolexpr{ test {\iffieldequalstr{entrysubtype}{classical}}
               and not test {\iffieldundef{shorttitle}} }
    {\ifciteseen
       {\blx@ibidreset\clearname{labelname}}
       {\savefield{title}{\cbxtitle}\restorefield{labeltitle}{\cbxtitle}}}
    {}}
\makeatother


%%%%%%%%%%%%%%%%%%%%%%%%%%%%%%%%%BIBLIO%%%%%%%%%%%%%%%%%%%%%%%%%%%%%%%%%%%%%
%Attention au chemin absolu ici !
\subfile{/Users/Matt/Desktop/Pre-these/Projet_de_these/Sortie_LaTeX/Bibliographie/Biblio_projet_de_these.tex}
\addbibresource{\jobname.bib}
\title{Bibliographie}
%\usepackage[urlcolor=green]{hyperref}%r\UTF{00E9}f\UTF{00E9}rences internes au document (toc, etc) \UTF{00E0} mettre en dernier


% Attention � l'�chappement des caract�res sp�ciaux, dont # et _ qui en font partie !
\usepackage{hyperref}
%Attention : fonctionnement: \href{A=cible(url ou r�f�rence interne)}{B=texte que vous voulez voir appara�tre} 
%A et B sont OBLIGATOIRES !!
% Attention � l'�chappement des caract�res sp�ciaux, dont # et _ qui en font partie !





\title{\textit{La version B du }Regimiento de los pr�n�ipes\textit{ glos�
               (1374-1494). �tude et �ditions de la partie sur le gouvernement de la
               cit� par temps de guerre (III, 3)}\vspace{-6ex}}\begin{document}\auteur{Matthias Gille Levenson\\Candidat � un CDSN}
       {\let\newpage\relax\maketitle}%voir https://tex.stackexchange.com/questions/86249/maketitle-text-before-title
       \legende{Projet de th�se}
   
   
      
         \section{I - �tat de la recherche}
            
            \subsection{Les travaux sur le De Regimine principum}
               
               
            
        
                   Le \textit{De regimine principum}
                  est une des oeuvres de litt�rature politique latine les plus diffus�es
                  au bas-Moyen �ge: plus de 350 manuscrits sont recens�s pour la seule
                  version latine [\cite[255]{briggs_manuscripts_1993}], et l'on compte des traductions
                  m�di�vales dans de nombreuses langues europ�ennes\footnote{Il n'existe
                     malheureusement pas � ce jour d'�dition du texte latin, mais
                     plusieurs des traductions au vernaculaire ont �t� �dit�es
                     ces vingt derni�res ann�es.}. �crit autour de 1280 [\cite{del_punta_egidio_1993}] par l'augustin
                  Gilles de Rome pour le futur Philippe le Bel, cet ouvrage est compos� de plus
                  de deux cent chapitres, et se compose de trois parties. Se gouverner soi,
                  gouverner sa maisonn�e et gouverner son royaume sont les th�mes d'un
                     \textit{speculum principum} qui sera un des vecteurs principaux de
                  diffusion de l'aristot�lisme m�di�val ainsi que de la conception
                  tripartite de la philosophie pratique (en �thique, �conomie et
                  Politique) en Europe [\cite[55]{bizzarri_fray_2000}]. Il est connu assez t�t en Espagne: don
                  Juan Manuel, par exemple, le mentionne dans le \textit{Libro enfinido},
                  autour de 1336-1337 [\cite[56]{bizzarri_fray_2000}]. 
               
            
         Le \textit{De regimine principum} est traduit au castillan autour de 1345,
                  ce qui contribue � sa forte diffusion en Castille entre les
                        \textsc{xiv}\textsuperscript{e} et \textsc{xv}\textsuperscript{e}
                  si�cles. Il sera par exemple r�sum� et synth�tis� en
                  partie par le lettr� Pedro de Chinchilla en 1464, dans un court ouvrage
                     (\textit{Exorta�i�n o ynforma�i�n de buena et sana
                     doctrina}) � destination du jeune ``roi-usurpateur''
                  Alfonse \textsc{xii}, texte que j'ai �dit� dans le cadre de mon
                  m�moire de Master I. L'autorit� du texte est probl�matique, tant
                  les divergences sont grandes entre les t�moins  [\cite{diez_garretas_juan_2002}]: la critique, ancienne comme plus r�cente, a
                  retenu Juan de Castrojeriz comme auteur du texte, mais la r�alit� de
                  cette autorit� est douteuse, ainsi que l'indiquent D�ez Garretas  [\cite[135]{diez_garretas_juan_2002}] et Mart�n Sanz [\cite[p. 199, note
                        n�6]{martin_sanz_magister_2009}]: Juan Garc�a de Castrojeriz semble �tre
                  pour les chercheurs actuels une autorit� de convenance. L'int�r�t
                  de la recherche de l'auteur de la recomposition est certain: il s'agit de
                  d�terminer le contexte id�ologique en particulier de re-cr�ation de
                  l'oeuvre de l'augustin. Des indices laissent ainsi penser que l'auteur est
                  franciscain (on retrouve ainsi des �l�ments qui font penser au
                  d�bat sur la pauvret� qui avait lieu chez les Franciscains [\cite[3-5]{briguglia_lost_2011}]).
                  En ce qui concerne les sources de cette traduction glos�e et remani�e,
                  on note un recours important, du fait de la longueur du texte, � des sources
                  ant�rieures/ext�rieures. Cependant, les sources se r�duiraient
                  � deux ou trois textes qui fournissent le plus gros de la mati�re de la
                  glose: on citera notamment Jean de Salisbury (avec le Policraticus), Jean de
                  Galles, source principale [\cite[6]{briguglia_lost_2011}], et Guillaume de Conches  [\cite{guardiola_influencia_1985}].
               
            
         Ce texte a connu une seule �dition acad�mique, publi�e en 1947 par
                  Juan Beneyto P�rez sous le titre \textit{Glosa castellana al ``Regimiento de
                     pr�ncipes''} et re-publi� en 2005 par l'\textit{Instituto de
                     Estudios Pol�tico} [\cite{juan_beneyto_perez_glosa_1947}]. Cette �dition est critiquable du
                  point de vue philologique\footnote{ \cite[2]{briguglia_lost_2011}. Juan Beneyto P�rez affirme
                     s'�tre servi de quelques manuscrits et de l'incunable pour composer son
                     �dition, mais ne propose aucun apparat critique ni note venant expliquer
                     sa d�marche d'�dition, par exemple. Le t�moin suivi est
                     majoritairement l'incunable de 1494.}, mais elle a le m�rite
                  d'exister et de proposer une vue d'ensemble approximative du texte. Un groupe de
                  chercheurs de l'Universit� de Valladolid travaille sur la traduction
                  glos�e au \textit{De regimine principum} depuis la fin des ann�es
                  1990. Ce groupe de chercheurs a d�termin� l'histoire globale du texte.
                  Selon eux, il est possible de mettre en lumi�re l'existence de deux
                  recompositions successives. Une premi�re recomposition donnerait lieu �
                  un texte dont la glose prend de plus en plus de place au d�triment de la
                  traduction, jusqu'� la faire dispara�tre compl�tement dans certains
                  chapitres. Une seconde recomposition voit le jour durant le
                     \textsc{xv}\textsuperscript{e} si�cle, remaniement qui a donn� un
                  texte r�duit � 104 chapitres, la structure en livres et parties ayant
                  disparu. Ces trois versions du texte, qui sont le fruit de remaniements successifs
                  mais dont les t�moins peuvent �tre contemporains, sont nomm�es A, B
                  et C\footnote{ \cite{carlos_alvar_glosa_2002}; \cite{diez_garretas_manuscritos_2003}; \cite{diez_garretas_transmision_2003}; \cite{diez_garretas_aproximacion_2004}; \cite{diez_garretas_versiones_2005}.}.Trois chercheurs sont encore
                  actifs sur ce th�me aujourd'hui: une professeure d'Universit�,
                  Mar�a Jes�s D�ez Garretas, et deux doctorants, dont Demetrio
                  Mart�n Sanz. Ces chercheurs travaillent sur les versions A et C: c'est de la
                  version B, qui ne conna�t donc pas d'�dition scientifique, que je vais
                  m'occuper. La comparaison de ce texte copi�, recopi�, glos� et
                  annot� � un champ de bataille textuelle autant que culturelle et
                  id�ologique est r�guli�re\footnote{ \cite{rodriguez_velasco_``bibliotheca_2001}; \cite{martin_sanz_magister_2009}; \cite[251]{rodriguez_velasco_produccion_2010}.}: plusieurs chercheurs ont mis en valeur la
                     ``bataille'' qui aurait lieu entre ces deux entit�s textuelles,
                  glose et traduction, que ce soit tant du point de vue du contenu du texte (le
                  volume compar� des deux entit�s) que de celui de la mise en forme, avec
                  des ph�nom�nes de disparition, de reconfiguration de la fronti�re
                  entre glose et traduction, puis de r�apparition de cette distinction avec le
                  manuscrit humaniste du \textsc{xv}\textsuperscript{e} si�cle (Francisco
                  Bautista). 
            
            \subsection{�tat de la recherche personnel}
               
               
            
        La version B, qui compte le plus de t�moin et est celle qui fut le plus
                  diffus�e jusqu'� la fin du \textsc{xv}\textsuperscript{e} si�cle n'a
                  pas �t� �tudi�e ni �dit�e comme telle: il
                  para�t donc n�cessaire et important de proposer un travail scientifique
                  dessus; le format de la th�se de doctorat peut bien s'y pr�ter. C'est la
                  raison pour laquelle je me suis int�ress� au texte: j'ai commenc�
                  � travailler sur le \textit{Regimiento de los pr�n�ipes} en
                  Master II. Il s'est agi pour moi de d�couvrir le texte selon une perspective
                  traditionnelle, � savoir une �dition critique de type b�dieriste,
                  avec comme base de travail une �dition au format XML-TEI/P5. Cela m'a permis
                  de tracer les premiers contours d'un arbre de relations probables entre manuscrits
                     (\textit{stemma codicorum}), et de proposer une �dition du texte
                  sur la premi�re partie du premier livre\footnote{\href{http://perso.ens-lyon.fr/matthias.gille-levenson/Annexes/pdf/Assemble.pdf}{http://perso.ens-lyon.fr/matthias.gille-levenson/Annexes/pdf/Assemble.pdf}}. J'ai aussi pu commencer une �tude id�ologique du texte, portant
                  notamment sur des questions de pauvret� ou de rapport entre pouvoir temporel
                  et spirituel\footnote{Un fait remarquable: le texte, a priori destin� � des
                     gouvernants pr�sents ou futurs, pr�sente une aversion surprenante
                     autant que cat�gorique pour le pouvoir et son caract�re corrupteur:
                           ``\textit{Noble grado m�s que bien andan�a, e si los
                           omnes bien supiesen qu�ntos son los cuydados e los peligros e las
                           mezquindades deste grado e desta corona, � non la dev�a ninguno
                           levantar de la tierra, mas dexarla yazer!}'' I, 1,
                     8.}: le franciscanisme ne semble jamais loin, comme l'affirme
                     Lambertini [\cite{briguglia_lost_2011}]. Mon travail
                  d'�dition a donn� deux r�sultats: une �dition traditionnelle
                     ``papier'', et une �dition au format html\footnote{On peut
                     retrouver cette �dition � l'adresse suivante: \href{http://perso.ens-lyon.fr/matthias.gille-levenson/Annexes/XSLT/Build/Edition-dig.html}{http://perso.ens-lyon.fr/matthias.gille-levenson/Annexes/XSLT/Build/Edition-dig.html}.}. En outre, j'ai commenc� � d�velopper un prototype
                  d'�dition synoptique des diff�rents prologues -- tous tr�s
                  divergents les uns par rapport aux autres -- des manuscrits du Regimiento\footnote{On
                     peut trouver cette �dition � l'adresse suivante: \href{http://perso.ens-lyon.fr/matthias.gille-levenson/etude-prologues/Prologues.html}{http://perso.ens-lyon.fr/matthias.gille-levenson/etude-prologues/Prologues.html}.}. Je compte m'appuyer sur ces premi�res bases pour commencer ma
                  th�se. D'un point de vue plus pratique, j'ai aussi pu me procurer une partie
                  des manuscrits que je vais �tudier pour ma th�se ayant travaill�
                  pour mon m�moire de Master II sur une partie des t�moins de mon corpus :
                  la majorit� des reproductions n'est pas librement disponible, mais est
                  cependant facilement accessible: les manuscrits sont tous en Espagne dans des
                  biblioth�ques publiques, � l'exception d'un t�moin qui se trouve
                  � Philadelphie, aux �tats-Unis.
            
         
         \section{II - Probl�matiques - Axes de recherche}
            
            \subsection{D�limitation du corpus}
               
               
            
        Plusieurs options s'offrent au chercheur qui veut travailler sur cette traduction
                  glos�e et recompos�e. Une �dition de la version B reste �
                  faire, si l'on consid�re que c'est un texte en soi, distinct de la
                  premi�re version, ce qui est le point de vue de la recherche actuelle. Cette
                  version B a �t� la plus diffus�e � partir du
                        \textsc{xv}\textsuperscript{e} si�cle, et c'est celle sur laquelle
                  travaille Beneyto P�rez en 1947. Une r�-�dition compl�te du
                  texte qui prendrait en compte sa nature de "seconde main" me semble serait donc un
                  travail justifi� du point de vue scientifique. Ce projet est
                  int�ressant, mais ce n'est pas celui que j'ai retenu pour ma th�se. En
                  effet, au vu de la taille de l'ouvrage et du corpus de manuscrits (plus de deux
                  cent chapitres, un dizaine de t�moins, entre 160 et 600 folios par
                  manuscrit), il semble assez clair que c'est un travail qui ne peut �tre
                  accompli en trois ou quatre ans, temps attendu de nos jours pour mener � bien
                  un projet doctoral complet; d'o� l'id�e de r�duire la taille du
                  corpus pour m'int�resser � la fin du texte, � savoir la partie 3 du
                  livre III. Son titre est ``\textit{Del gobierno de la ciudad e del reyno en
                        tiempo de guerra}''. C'est la deuxi�me partie d'un
                  diptyque qui traite du gouvernement de la Cit�, la premi�re �tant
                  consacr�e au gouvernement en temps de paix\footnote{Segunda parte,
                           ``\textit{Del gobierno de la ciudad e del reyno en tiempo de
                           paz}''. La premi�re partie du troisi�me livre
                     est consacr�e aux ``opinions des philosophes''(je
                     traduis).}. Cette derni�re partie est tr�s homog�ne sur le
                  fond, elle traite de la question de la noblesse et plus particuli�rement de
                  la chevalerie; son �tude me permettra � la fois de travailler sur
                  l'histoire mat�rielle globale du texte et sur des aspects id�ologiques
                  plus pr�cis. Le corpus choisi s'�tend donc sur III, 3, soit la
                  derni�re partie de l'ouvrage, qui compte vingt trois chapitres. En ce qui
                  concerne les t�moins de mon travail, je choisirai les manuscrits de B qui
                  courent sur tout le livre III, � savoir un total de six manuscrits plus un
                  incunable.
               
            
            \subsection{Noblesse et chevalerie}
               
               
            
        L'�tude du texte que je me propose de faire s'articulera autour du concept de
                  chevalerie. Plusieurs chercheurs �tudient depuis une vingtaine d'ann�es
                  les questions de chevalerie dans la Castille m�di�vale; le concept de
                  chevalerie est difficile � s�parer de celui de noblesse, tant les deux
                  sont entrem�l�s, mais il ne lui est cependant pas �quivalent. Il me
                  semble important d'insister sur le flou et la variabilit� de l'id�e de
                  chevalerie, sur la difficult� qu'il y a � d�finir ce concept. Comme
                  l'affirme Carlos Heusch, ``\textit{la realidad caballeresca es, en su mayor
                        parte, un mundo de textos, de discursos}'' [\cite[16]{heusch_caballericastellana_2000}]: il
                  est soumis � l'interpr�tation des auteurs/acteurs des textes et de leurs
                  int�r�ts id�ologiques autant que politiques. En Castille, au bas
                  Moyen-�ge, le chevalier, qui peut �tre d�fini \textit{a
                     minima} comme ``l'homme d'armes mont�'' [\cite{gauvard_dictionnaire_2002}], va
                  n�cessiter un appui th�orique et juridique que vont apporter des acteurs
                  comme le pouvoir monarchique. Le concept de chevalerie surgit presque \textit{ex
                     nihilo} comme concept politique au \textsc{xiii}\textsuperscript{e}
                  si�cle, bien qu'ayant probablement pr�exist� en tant qu'ordre
                  social: il s'agit, selon Rodr�guez Velasco, d'une ``invention'' de
                  la Monarchie, au sens rh�torique du terme  [\cite[XI-XII]{fleckenstein_caballeriy_2006}]. On observe ainsi deux mouvements parall�les
                  concernant la chevalerie, th�orisation et
                  l�galisation.
                  La chevalerie n'est pas n�cessairement noble au
                     \textsc{xiii}\textsuperscript{e} ni au \textsc{xiv}\textsuperscript{e}
                  si�cles, comme le prouve l'existence des \textit{caballeros
                     villanos} [\cite[11]{heusch_caballericastellana_2000}]: l'assimilation de la
                  chevalerie � la noblesse se r�alise dans ces deux si�cles de fin du
                  Moyen-�ge (cela arrive plus t�t pour le royaume de France, autour de la
                  fin du \textsc{xiii}\textsuperscript{e} si�cle, selon Armand Arriaza [\cite[338]{arriaza_noblesse_2006}]); elle commence avec les \textit{Partidas}, texte dans
                  lequel on a not� l'absence de r�f�rences � la chevalerie non
                     noble [\cite[11-12]{heusch_caballericastellana_2000}]. Dans ce bas Moyen-�ge, la chevalerie
                  est un enjeu de pouvoir dans l'�quilibre des forces entre noblesse et
                  monarchie. Ainsi, au \textsc{xiv}\textsuperscript{e} si�cle, il s'agit pour le
                  pouvoir monarchique d'englober la noblesse dans une chevalerie qui a pour
                  t�te le roi; le renforcement de la noblesse a alors pour fonction le maintien
                  d'un ordre politique et social qui met au centre du jeu le Roi dans un contexte de
                  tensions politiques entre la monarchie et la noblesse\footnote{ \cite{rodriguez_velasco_oficio_1993}; \cite[63]{heusch_caballericastellana_2000}.}. Appara�t � ce moment ce
                  que l'on a pu nommer l'``image chevaleresque de la monarchie'' [\cite[35]{rodriguez_velasco_diego_1996}].
                  Jes�s Rodr�guez Velasco a mis en valeur trois �tapes principales
                  dans l'histoire de la chevalerie en Castille: une p�riode de d�finition
                  juridique, avec les \textit{\textit{Partidas}} notamment
                  (1250-1350), une de restriction (1350-1390), o� l'on voit appara�tre la
                  chevalerie comme distinction et privil�ge accord� par le Roi (c'est le
                  sens de l'Ordre de la \textit{Banda}). En relation avec ces enjeux de
                  pouvoir, une des particularit�s du texte de Gilles de Rome, selon
                  Rodr�guez Velasco, est qu'il accorde une place particuli�re � la
                  vertu de prudence pour d�finir la chevalerie. Gilles de Rome serait ainsi le
                  premier � avoir th�oris� et construit une hi�rarchie des
                  prudences, le Roi �tant le seul � poss�der tous les types de
                     prudence\footnote{Les cinq types de prudence sont, en castillan, les suivants:
                        \textit{singular}, \textit{ycon�mica},
                        \textit{regnativa}, �\textit{ibdadana},
                        \textit{cavalleresca}.}; pour le chevalier, cette prudence
                  devient ``la vertu par antonomase'' (je traduis l'expression du
                  chercheur), qui �tait jusqu'alors la force d'�me. Cette �volution
                  s'inscrit dans un mouvement de l�gitimation des lettr�s qui peuvent
                  ainsi pr�tendre de fa�on plus l�gitime au statut de chevaliers [\cite[23-24]{rodriguez_velasco_diego_1996}]. On observe enfin une derni�re
                  �tape d'expansion (1390-1492) du d�bat sur la chevalerie, avec plus de
                  80 trait�s consacr�s � la question [\cite{rodriguez_velasco_para_1997}]. C'est �
                  cheval entre la seconde et la troisi�me �tape que sont copi�s et
                  diffus�s nos t�moins.
                  
               
            
        
               
            
         Le rapport de ces concepts et statuts au bartolisme, courant juridique du nom de
                  Bartole de Sassosferrato (1313-1356), qui semble lui-m�me avoir �t�
                  influenc� par Gilles de Rome [\cite[1]{briguglia_lost_2011}], est particuli�rement important: en
                  Castille, une pragmatique de 1427, date qui correspond au moment de plus forte
                  circulation et diffusion des manuscrits que j'�tudie, �dicte
                  l'obligation de suivre la doctrine de Bartole\footnote{ \cite[114]{rodriguez_velasco_diego_1996}; \cite[36-39]{rodriguez_velasco_discurso_2000}.} en ce qui concerne la
                  noblesse. Plus g�n�ralement, circulent � cette �poque en
                  Europe une ensemble de textes et d'id�es autour de la noblesse et la
                  chevalerie qui ont eu une influence importante en Espagne. Selon Heusch et
                  Rodr�guez Velasco, la nouveaut� des id�es de Bartole r�side
                  dans le fait qu'elle met en avant la vertu individuelle du chevalier  [\cite[14]{heusch_caballericastellana_2000}]: on retrouve ici de forts �chos avec le
                        \textit{De Regimine} qui met lui-m�me en avant la
                  vertu du prince et plus g�n�ralement de tout lecteur
                           potentiel\footnote{``\textit{La ter�era es que maguer este libro
                           se faga para los Reyes, enpero todos los omnes pueden ser ense�ados
                           por �l, \& por ende todos lo deven aprender \& saber, \&
                           �ierto es que el pueblo non puede ser tan s�til que pueda
                           aprender Razones s�tiles, \& por ende conviene que se den en
                           �l Razones gruesas \& palpables \& exenplos muchos de los
                           Reyes \& de los omnes, por que los puedan todos
                        aprender.}''
                     \textit{Regimento de los pr�n�ipes}, I, 1, 1.}. Ce
                  courant juridique a donc connu une r�ception particuli�rement favorable
                  aupr�s des pouvoirs centraux europ�ens. Cela peut ais�ment
                  s'expliquer, selon Armand Arriaza, si l'on se rappelle les enjeux de pouvoir entre
                  noblesse et royaut� que j'ai �voqu�s plus haut: ``ce qui
                     attirait l'attention des souverains de toute l'Europe, c'est que dans le
                     tra�t� [de Bartole], le pouvoir d'octroyer un statut nobiliaire
                     �tait du ressort exclusif de l'autorit� centrale d'une entit�
                     politique autogouvernante'' [\cite[346-347]{arriaza_noblesse_2006}]. 
            
            \subsection{Variance et New Philology}
               
               
            
         Pour comprendre les axes d'�tude que j'ai choisis pour �tudier la
                  traduction glos�e au \textit{De regimine}, il me semble
                  important de revenir sur un d�bat m�thodologique et de pr�senter
                  l'approche d'�tablissement des textes que j'ai choisi de suivre. La
                     ``nouvelle vague'' critique qu'on a appel� Nouvelle Philologie
                  (aussi appel�e \textit{Material Philology}) appara�t au
                  d�but des ann�es 1990\footnote{Le volume n�65 de Speculum, paru en
                     janvier 1990, est consid�r� comme le point de d�part de la
                        \textit{New Philology}. Cette d�nomination de ``Nouvelle
                        Philologie'' pose elle-m�me d�bat: voir  \cite[113]{frandsen_dialectique_2005}.}. Elle est inspir�e\footnote{Comme le
                     montre l'article introductif de la revue Speculum par Nichols d�s la
                     premi�re page. \cite[1]{nichols_introduction:_1990}.} d'une s�rie d'articles et
                  d'ouvrages publi�s dans les ann�es 1980 par Bernard
                        Cerquiglini\footnote{ \cite{cerquiglini_eloge_1983}; \cite{cerquiglini_eloge_1989}.}, qui posent que le texte
                  m�di�val est fondamentalement variance \footnote{``L'oeuvre
                        litt�raire, au Moyen Age, est une variable'':  \cite[57]{cerquiglini_eloge_1989}
                  .}. Cette ``philologie nouvelle'' tente de proposer une
                  alternative � l'opposition historique entre (n�o)lachmannisme et
                  b�di�risme, et notamment refuse l'id�e d'�ditions fond�es
                  sur sur la hi�rarchisation des t�moins en fonction de leur valeur  [\cite[140-141]{stolz_new_2003}]. En l'id�alisant, les deux m�thodes
                  passeraient � c�t� du texte m�di�val qui est justement
                  absence d'unicit� [\cite[33]{cerquiglini_eloge_1983}]: au contraire, la New Philology
                  met l'accent sur la mat�rialit� du manuscrit, r�sultat d'une
                  multitude d'op�rations intellectuelles comme artistiques ou techniques [\cite{nichols_introduction:_1990}]. L'arriv�e
                  de ce courant philologique a donn� lieu � des pol�miques
                  importantes, comme l'article de Cerquiglini (2000) le fait sentir\footnote{ \cite{cerquiglini_nouvelle_2000}. La lecture des premi�res
                     pages de Varvaro 1999 suffit � se rendre compte des remous qu'ont
                     suscit� les �crits de Cerquiglini, au ton souvent provocateur:  \cite{varvaro_new_1999}.}. D'un point de vue historique et
                  technique, un point d'achoppement entre les scientifiques qui est � relever
                  semble �tre la qualit� du scribe, entre copiste et remanieur, si l'on
                  veut une opposition sch�matique et pour reprendre les mots de Bernard
                     Cerquiglini [\cite[30]{cerquiglini_eloge_1983}]; plus g�n�ralement,
                  Matthew Driscoll propose une distinction int�ressante faite entre oeuvre
                        (``\textit{work}'', je traduis), texte et artefact: la
                     \textit{New Philology} mettrait l'accent sur le dernier
                  �l�ment, le plus mat�riel, tandis que la philologie traditionnelle
                  tendrait plus � privil�gier le texte avec comme objectif d'arriver
                  � l'oeuvre [\cite{driscoll_words_2010}]. Le
                  d�bat porte ici surtout sur des textes po�tiques, au sens le plus fort
                  du mot; dans le cas qui m'int�resse, c'est plut�t une opposition entre
                  �coles id�ologiques, entre familles, statuts, classes sociales ou
                  simplement moments historiques divergents qui seront au centre de mon travail. En
                  ce sens, consid�rer chaque manuscrit comme un t�moin � part
                  enti�re qui peut diverger des autres me semble �tre un point de vue
                  productif et tr�s int�ressant, avec les limites philologiques que l'on
                  peut imaginer, quand l'on se trouve par exemple en pr�sence de clairs
                     \textit{codices descripti}. Les id�es que j'avance me semblent
                  particuli�rement adapt�es aux manuscrits que j'�tudie et �
                  l'histoire de leur texte; je n'ai pas de volont� particuli�re de
                  th�oriser de fa�on g�n�rale, globale, ni, comme je l'ai dit
                  plus haut, de me positionner de fa�on abstraite dans le d�bat
                     philologique\footnote{Ainsi, selon Frandsen, on ne peut parfois �chapper
                     � la notion d'oeuvre ni d'intention de l'auteur, comme c'est le cas par
                     exemple avec le Chansonnier N. \cite[117]{frandsen_dialectique_2005}.}: je choisis
                  de privil�gier une m�thode qui me semble particuli�rement
                  appropri�e � l'�tude de mes t�moins. Mon travail s'inscrira
                  donc principalement dans ce courant m�thodologique; il ne s'agit cependant
                  pas de suivre une �cole mais bien d'utiliser un ensemble d'outils conceptuels
                  et th�oriques pour travailler sur les t�moins du
                        \textit{Regimiento}. Les caract�ristiques des textes
                  face auxquels l'on se trouve, plusieurs �tats de texte, un flou auctorial
                  certain, se pr�tent bien � une interpr�tation qui suivrait les
                  id�es fortes de la \textit{New
                  Philology}.
            
            \subsection{Les actualisations du texte}
               
               
            
        Mon travail ne sera pas une �tude de la noblesse ni de la chevalerie aux
                        \textsc{xiv}\textsuperscript{e} et au \textsc{xv}\textsuperscript{e}
                  si�cle, deux questions auxquelles ont �t� consacr�e deux
                  th�ses au cours des vingt derni�res ann�es\footnote{ \cite{rodriguez_velasco_diego_1996}; \cite{gonzalez_vazquez_representation_2013}.}, mais bien une �tude
                  du corpus des diff�rents \textit{Regimiento de los pr�n�ipes}
                  au prisme des questions de noblesse et de chevalerie. Par rapport � la
                  publication de 1947, il s'agira de proposer une �dition claire et
                  document�e de plusieurs t�moins sur III, 3; plus avant, il s'agira de
                  d�terminer les points de fluctuation entre chaque t�moin et entre les
                  familles. Je consid�re ainsi que les m�thodes et la conception du
                  manuscrit envisag�es par la ``philologie nouvelle'' [\cite{cerquiglini_nouvelle_2000}] peuvent
                  �tre un axe int�ressant d'�tude exp�rimentale du texte. Le
                  genre du texte qu'il m'est donn� d'�tudier, un miroir des princes, un
                  texte avant tout th�orique -- m�me si la mati�re narrative ne peut
                  pas �tre mise de c�t� -, se pr�te particuli�rement �
                  l'id�e de variation selon le contexte de production et de copie des textes.
                  L'id�e globale est donc la suivante pour l'�tude de mon texte: aller
                  chercher dans la variation entre les diff�rents t�moins des points de
                  friction qui seraient significatifs du point de vue de la th�orie, de
                  l'id�ologie que l'on veut transmettre, des �l�ments
                        ``\textit{sociologically and historically
                     interesting}'' pour reprendre Driscoll [\cite{driscoll_words_2010}]. Ces points de friction semblent clairs entre les
                  manuscrits et l'incunable (� tel point que l'on pourrait presque m�me
                  parler d'une nouvelle recomposition du texte\footnote{L'incunable, qui est
                     rattach� � la version B, montre de fa�on assez explicite que la
                     question m�rite d'�tre pos�e: il est assez diff�rent par
                     rapport aux manuscrits de la m�me version. Ajouts, remaniement de la
                     structure (des chapitres sont scind�s en III,3, par exemple) sont
                     ais�ment observables pour ce t�moin, ce qui est par ailleurs une
                     bonne preuve que l'�dition acad�mique de 1947, fond�e
                     essentiellement sur l'incunable, est � refaire.}), plus subtils
                  entre les diff�rents manuscrits. Je refuse l'id�e d'une
                  d�t�rioration des textes avec le temps: pour l'instant, il me
                  para�t bien plus profitable de parler d'actualisations successives [\cite[8]{nichols_introduction:_1990}].
               
            
        La premi�re question � se poser est tr�s g�n�rale :
                  quelle est exactement l'histoire de ce texte, si important si l'on prend en compte
                  sa diffusion au bas Moyen-�ge? Je pr�ciserai cette question par le biais
                  des interrogations suivantes: comment fonctionne la seconde version par rapport
                  � la premi�re? Quelles sont les caract�ristiques de la
                  recomposition qui a eu lieu? Qu'a-t-on voulu faire du \textit{Regimine
                     principum} aux \textsc{xiv}\textsuperscript{e} et
                     \textsc{xv}\textsuperscript{e} si�cle? La question de la r�ception du
                  texte m�di�val est ici centrale. Le rapport entre traduction et glose,
                  avec une affirmation toujours plus importante de la derni�re au fil du temps,
                  sera particuli�rement �tudi�. Cet ensemble de questions sera
                  pos� en regardant � travers le prisme de l'�volution
                  id�ologique du texte: l'actualisation formelle du \textit{Regimiento de los
                     pr�n�ipes glosado}, la plus visible, a-t-elle �t�
                  accompagn�e d'une actualisation de la pens�e du texte sur la noblesse et
                  la chevalerie? Il semble acquis qu'il y a au moins deux textes entre A et
                     B\footnote{Il est important de ne pas penser ces versions en termes de succession
                     chronologique: bien que logiquement cr��es l'une apr�s l'autre,
                     elles se diffusent de tout �vidence simultan�ment en
                  Castille.}; je voudrais pouvoir aller plus loin et d�terminer, pour le
                  dire synth�tiquement, le caract�re homog�ne ou
                  h�t�rog�ne de la version B . Plus g�n�ralement, je
                  propose la probl�matique suivante: \textit{quelles sont les
                     marques mat�rielles et textuelles du d�bat et plus
                     g�n�ralement de l'histoire des concepts de noblesse et de chevalerie
                     pr�sentes dans les t�moins du ``Regimiento de los
                        pr�n�ipes glosado''?} Ces marques apparaissent-elles au
                  travers de variance entre les t�moins de la version B? � partir des
                  r�ponses apport�es � cette premi�re question, je pourrai
                  tenter de formuler une r�ponse plus g�n�rale et plus th�orique
                  au d�bat philologique qui a cours depuis les ann�es 1990, comme
                  pr�sent� plus haut: il me sera possible de d�terminer s'il y a
                  lieu, du point de vue id�ologique et pour le corpus singulier que
                  j'�tudie, de parler de ``texte'' au pluriel\footnote{Par
                     commodit� de langage, je tends dans cette pr�sentation � parler
                     de texte au singulier.}.
               
            
        La
                  noblesse et la chevalerie seront donc l'axe principal de mon �tude, mais je
                  m'int�resserai aussi � un aspect important du texte que constitue
                  l'ensemble des \textit{exempla} dont il est compos� (cela
                  correspond en volume � une bonne moiti� du texte), fragments que
                  j'�tudierai en tant qu'objets narratifs\footnote{La place que prend cette forme
                     narrative est en effet tr�s int�ressante: par exemple, elle en vient
                     presque � devenir ind�pendante de la th�orie expos�e par le
                     texte au-fur-et-�-mesure de la progression de III, 3. En effet, les
                     derniers moments du texte exposent/racontent l'histoire d'Alexandre de
                     fa�on suivie, alors que la majorit� des exemples, jusque alors,
                     n'�taient g�n�ralement pas reli�s.}. Cette �tude
                  n'est cependant pas sans rapport avec l'objet principal de mon projet doctoral: la
                  mati�re exemplaire est d'un genre particulier, elle est presque exclusivement
                  historique (pour III, 3, nous ne nous trouvons donc pas devant des exemples
                  traditionnels), une caract�ristique qui s'inscrit pleinement dans le contexte
                  de naissance de l'humanisme castillan dans la noblesse de la fin du
                        \textsc{xiv}\textsuperscript{e}- d�but du \textsc{xv}\textsuperscript{e}:
                  un mod�le de chevalier humaniste est en train d'appara�tre mettant en
                  avant les mod�les antiques [\cite[64]{heusch_chevalerie_2000}]. Une �tude pr�cise
                  des \textit{exempla} pr�sents dans mon texte (dans les chapitres
                  choisis pour l'�dition ou, plus globalement, dans l'oeuvre enti�re) me
                  semble donc avoir sa place dans mon travail doctoral; pour ce, je m'appuierai sur
                  les travaux r�alis�s sur le texte que j'�tudie [\cite{diez_garretas_recursos_2009}] ainsi que sur les �tudes plus
                  g�n�rales sur l'\textit{exemplum}, celles de la recherche
                  fran�aise notamment [\cite{berlioz_les_1998}].
                  
            
            \subsection{M�thodologie}
               
               
            
        
                   Comme affirm� plus haut, il nous a
                  sembl� n�cessaire de restreindre mon travail � une partie du texte.
                  Je n'abandonne pas pour autant mon ambition d'explication globale du texte, tant
                  du point de vue philologique qu'id�ologique. On pourra ainsi voir mon travail
                  comme une �tude microscopique � vis�e macroscopique. Ce travail ne
                  sera donc pas lin�aire, et devra aborder des points qui me permettront de
                  comprendre le texte et sa construction dans leur ensemble. Ainsi, je pense
                  int�grer � mon travail de th�se l'�tude/�dition
                  synoptique des prologues, que j'aurai approfondie, et que j'ai commenc�
                  � r�aliser cette ann�e. Les prologues sont en effet, comme je l'ai
                  affirm� plus haut, tr�s divergents, et sont d'une grande aide pour mon
                  travail. Cette �tude est une premi�re id�e de ce que pourra
                  �tre mon �dition des versions du texte. La question de la distinction
                  entre glose et traduction est un point important et int�ressant: la
                  pr�sentation du texte r�sulte forc�ment d'un choix �ditorial
                  sur le point de vue adopt�: � quel point les contemporains des textes
                  avaient-ils conscience de la qualit� des textes qu'ils lisaient? Savaient-ils
                  diff�rencier ais�ment la glose de la traduction, pour des manuscrits
                  qui, dans leur extr�me majorit�, ne faisaient pas la diff�rence
                  entre les deux? 
               
            
        En ce qui concerne les m�thodes pr�cises d'�tude des textes, le
                  travail principal sera celui de la comparaison pr�cise et m�thodique des
                  t�moins sur III, 3, en prenant en compte tous les aspects mat�riels
                  � disposition pour tenter de d�terminer les usages des manuscrits en
                  plus de leur contenu. Cependant, un travail philologique plus classique peut
                  �tre int�ressant: une m�thode ne vient pas balayer l'autre [\cite{driscoll_words_2010}]. Il s'agira
                  d'approfondir ma connaissance des relations entre manuscrits: cela me para�t
                  �tre encore utile, non pas pour l'�dition et le classement des
                  t�moins, mais bien pour la compr�hension de l'histoire des textes ainsi
                  que de leur circulation. Cela dit, il me faudra �tre tr�s attentif aux
                  critiques que formule d�j� B�dier � la fin des ann�es
                  1920 sur les limites �pist�mologiques que supposent les
                     \textit{stemmata}\footnote{``Or, quand une fois [l'�diteur]
                        s'est arr�t� � l'�tape, qu'il croit la derni�re,
                        d'un classement en trois familles, \textit{x}, \textit{y}, \textit{z}, il ne se peut
                        gu�re qu'il ne rencontre quelques variantes unissant \textit{x} et \textit{y} contre (ou \textit{x} et \textit{z} contre \textit{y}, ou \textit{y} et \textit{x} contre \textit{z}) qui lui
                        sugg�rent l'id�e qu'elles peuvent repr�senter des
                        innovations, donc des fautes [...] la force dichotomique, une fois
                        d�cha�n�e, agit jusqu'au bout. Le syst�me lachmannien
                        l'a lanc� dans la chasse aux fautes communes, mais sans lui donner
                        aucun moyen de savoir � quel moment il a le devoir de
                        s'arr�ter.'' \cite[15-16]{bedier_tradition_1970}.}. Une
                  attention particuli�re sera accord�e � l'utilisation d'outils
                  informatiques pour comprendre les relations entre manuscrits, en m'appuyant sur la
                  recherche d�j�
                  effectu�e.
               
            
         Si l'on veut travailler sur de la variance, et
                  consid�rer le texte comme un t�moin (au sens presque juridique du
                  terme), il est indispensable de s'int�resser et de tenter de dater avec
                  pr�cision les manuscrits sur lesquels je vais travailler. Le travail
                  indiqu� au paragraphe pr�c�dent pourra poser un premier jalon, avec
                  une chronologie relative entre les t�moins. Le t�moin le plus
                  r�cent qu'est l'incunable de 1494, par sa grande variation, me para�t
                  particuli�rement
                  int�ressant:
                  une �tude des imprimeurs et des cr�ateurs de ce t�moin me semble
                  indispensable pour d�terminer la r�ception du texte dans l'Espagne des
                  Rois Catholiques, le texte ayant �t� imprim� � S�ville,
                  dans un contexte d'utilisation croissante de l'imprimerie par la monarchie dans un
                  but de propagande [\cite[Nieto Soria 1999: 299-301]{nieto_soria_origenes_1999}]. Plus largement, un
                  travail de mise en contexte historique sera n�cessaire, notamment sur la
                  question de la production, de la r�ception et de la transmission des textes.
                  Il faudra ainsi travailler sur les ordres qui semblent souvent �tre �
                  l'origine des textes, en particulier les ordres franciscains; une �tude des
                  centres de production et de copie (notamment nobiliaires au
                     \textsc{xv}\textsuperscript{e} si�cle, avec l'apparition et le fort
                  d�veloppement des biblioth�ques priv�es nobiliaires [\cite{beceiro_pita_libros_2007}]) pourront permettre d'�clairer l'histoire
                  des t�moins. De m�me, il sera important, dans mon �tude
                  mat�rielle des textes, de reprendre l'�tude h�raldique
                  r�alis�e en 2003 par F�lix Mart�nez Llorente [\cite[98-106]{diez_garretas_manuscritos_2003}] pour compl�ter mon travail sur le
                  contexte de production et de copie/re-cr�ation des textes. 
            
         
         \section{III - Un ensemble d'outils techniques � mettre en place}
            
            
            
        Pour des raisons de praticit� et de clart�, j'ai d�cid� de
               s�parer les aspects techniques et plus th�matiques dans la
               pr�sentation de mon projet de th�se; cependant cette distinction est
               artificielle tant les enjeux sont entrem�l�s. Mon travail sur les textes du
                     \textit{Regimiento}, tant philologique qu'informatique, est
               intimement li� � ma r�flexion m�thodologique: l'ordinateur me
               para�t �tre un outil tr�s utile pour le philologue. Cette relation
               entre informatique et philologie n'est pas nouvelle ni originale: l'outil
               informatique est d�j� consid�r� avec beaucoup d'int�r�t
               par la Nouvelle Philologie. Ainsi, gr�ce � l'�cran d'ordinateur et au
               caract�re dynamique de l'�dition �lectronique, la variance est ainsi
               plus facile � d�terminer et � pr�senter, me semble-t-il,
               gr�ce � l'ensemble des techniques modernes d'�dition dont je
               m'appr�te � faire une pr�sentation sommaire. Ce travail s'inscrit dans
               un cadre de pens�e qui est � rapprocher de la ``philosophie du
                  libre'' [\cite{vainio_free_2007}]: j'accorde une importance
               particuli�re � la question du partage de la connaissance, tant savante que
               technique. 
            \subsection{Une �dition �lectronique native. Le standard XML-TEI/P5.}
               
               
            
        Mon �dition du \textit{Regimiento de pr�n�ipes} glos�
                  sera ``nativement digitale'': elle sera fond�e sur un ensemble de
                  documents XML-TEI\footnote{Je reprends cette d�nomination du travail de
                     th�se (en cours) d'Ariane Pinche:\href{http://theses.fr/s150996}{http://theses.fr/s150996}. .}. La TEI (pour \textit{Text
                     Encoding Initiative}) est un consortium international de chercheurs en
                  humanit�s, fond�e � la fin des ann�es 80, et qui a pour
                  objectif la mise en place d'un standard de r�gles d'�ditions de textes
                  de tous types, en utilisant le langage de description XML\footnote{Une bonne
                     introduction � XML peut �tre trouv�e dans l'article de
                        Salvador \cite[101-105]{salvador_isilex_2017} Voir aussi: \href{https://fr.wikipedia.org/wiki/Extensible\_Markup\_Language}{https://fr.wikipedia.org/wiki/Extensible\_Markup\_Language} et \href{https://fr.wikipedia.org/wiki/Text\_Encoding\_Initiative}{https://fr.wikipedia.org/wiki/Text\_Encoding\_Initiative}.}. Ces r�gles ou recommandations sont nomm�es en anglais
                     \textit{Guidelines}\footnote{\href{http://www.tei-c.org/Guidelines/}{http://www.tei-c.org/Guidelines/}.}; ce sont celles que je
                  suis dans mon travail. Je trouve deux int�r�ts majeurs et fondamentaux
                  � l'utilisation et la pratique de la norme d'�dition de textes
                  XML-TEI/P5. Le premier concerne la description id�ale (au sens presque
                  platonicien) des textes. Un encodage TEI id�al permet une description du
                  texte dans ce qu'il est, sans aucune interf�rence avec la mani�re dont
                  il sera pr�sent� dans l'�dition. En quelque sorte, un fichier
                  XML-TEI propose ``un texte sans forme'', l'id�e du texte (ou du
                  manuscrit, ou de l'oeuvre, suivant la vision envisag�e) pure. Le second
                  int�r�t de la norme TEI/P5 r�side � mes yeux dans son
                  caract�re standard: son ambition id�ale est ainsi de voir toutes les
                  �ditions scientifiques r�alis�es selon les m�mes normes, avec
                  les m�mes r�gles et pouvant �tre process�es de la m�me
                  mani�re, pour pouvoir garantir une interoperabilit� maximale, ainsi
                  qu'une facilit� de mise en place de travaux collaboratifs, comme par exemple
                  avec le projet d'�dition collaborative et ouverte de testaments de poilus,
                  mis en ligne il y a peu\footnote{\href{https://testaments-de-poilus.huma-num.fr}{https://testaments-de-poilus.huma-num.fr}.}. Ainsi par
                  exemple, une �dition XML-TEI bien form�e (c'est-�-dire, conforme
                  � la syntaxe fondamentale du XML) et valide (conforme aux r�gles
                  �dict�es par la TEI) pourra �tre r�utilis�e dans d'autres
                  projets, que son auteur n'avait peut �tre pas imagin�s: on pourra
                  int�grer les textes �dit�s dans une base de donn�es par
                  exemple, un dictionnaire, ou autre. Le travail collaboratif en est simplifi�,
                  l'utilisation d'outils num�riques, de programmes informatiques ou
                  d'algorithmes pour traiter des textes, facilit�e. En ce sens, si l'on se
                  place dans la perspective plus large des humanit�s num�riques,
                  l'int�r�t de l'�dition �lectronique r�side selon moi
                  avant tout dans la possibilit� de standardiser r�ellement les textes (en
                  consid�rant comme horizon id�al un usage exclusif de la TEI par les
                  philologues). Les avantages d'une �dition �lectronique native sont
                  �videmment tr�s importants, et on retient en g�n�ral
                  l'id�e de pouvoir proposer plus que ce qu'un livre imprim� ou
                  num�ris� peut faire\footnote{Il faut cependant faire attention � ne pas
                     confondre le format de sortie avec le type d'�dition r�alis�e:
                     une �dition �lectronique native peut tout � fait donner comme
                     r�sultat un livre imprim�; elle offre cependant aussi des
                     possibilit�s plus larges.}; cependant je les consid�re comme
                  secondaires � c�t� de la r�elle r�volution scientifique
                  et �pist�mologique que constitue la cr�ation et le maintien d'un
                  standard de description de textes comme l'est la \textit{Text Encoding
                     Initiative}.
               
            
        En amont du travail d'encodage des textes, il sera important de r�aliser deux
                  op�rations qui engageront toute l'�dition. Ces deux op�rations,
                  intimement li�es, sont les suivantes: d�terminer, en premier lieu, quoi
                  encoder et comment (normes de transcription, niveau de d�tail de l'encodage
                  des textes). Je compte travailler au plus pr�s de la r�alit�
                  textuelle. J'indiquerai donc la ponctuation, les abr�viations, les sauts de
                  ligne, les marques textuelles de seconde main/relecture (ajouts, suppressions,
                  notes, passages surlign�s), l'ensemble des donn�es graphiques/textuelles
                  qui entourent le texte et le mettent en forme (lettrines, rubriques, titres
                  courants, mots d'appel, etc). Ce travail sera coupl� de la r�daction de
                  la documentation de mon travail �ditorial, par le biais d'un fichier XML
                  nomm� ODD\footnote{Il est possible de trouver plus d'information sur l'ODD ici
                        \href{http://www.tei-c.org/Guidelines/Customization/odds.xml}{http://www.tei-c.org/Guidelines/Customization/odds.xml}.},
                  pour ``\textit{One Document Does it all}'', document
                     tout-en-un [\cite[p. 93, traduction de Marjorie Burghart]{burnard_quest-ce_2015}]. Ce
                  document a deux utilit�s principales. En simplifiant beaucoup, il permet
                  d'abord une explication synth�tique du travail d'encodage [\cite[100]{burnard_quest-ce_2015}]], et apr�s transformation, donne un document (le
                  sch�ma) qui renferme de fa�on compr�hensible pour un ordinateur les
                  r�gles de la TEI, personnalis�es pour mon travail, ce qui permet de
                  travailler avec des gardes-fou. Le processus de v�rification de la
                  conformit� aux r�gles d�finies s'appelle la validation\footnote{Par
                     exemple, le logiciel saura me dire ``tu es pass� du folio 145v �
                        146v, attention!'', ou, de fa�on plus utile, ``tu ne
                        respectes pas les r�gles de la TEI, tu n'encodes pas de fa�on
                        standard: telle balise n'a pas sa place ici''.}). 
            
            \subsection{Valorisation du travail doctoral}
               
               
            
        La vision que j'ai de ma th�se d'�dition serait de pouvoir proposer un
                  panorama clair, bien que r�duit � la derni�re partie du livre III,
                  de l'histoire de cet ensemble de textes, avec la possibilit� pour le lecteur
                  de comparer les diff�rentes versions, les diff�rents t�moins, de
                  fa�on globale et autonome, mais aussi et avant tout guid�e\footnote{Dans le
                     cadre de ce travail, je voudrais pouvoir poser les bases d'un outil technique
                     de recherche et d'interrogation des exemples pr�sents dans mon texte,
                     point qui m'a d�j� int�ress� dans la premi�re de mes
                     deux petites �ditions digitales. Comme il m'a sembl� important de le
                     mentionner, la part d'exempla est importante dans l'ouvrage et je veux les
                     �tudier. Une mise en valeur de cette mati�re narrative me semble un
                     objectif important: j'ai comme objectif final la cr�ation d'une base de
                     donn�es qui rassemblerait ces exemples. Mon �dition html de la
                     premi�re partie du texte proposait un petit outil d'indexation que je
                     reprendrai dans mon �dition � venir. � plus long terme,
                     l'id�e est d'�tendre cette base de donn�es � tous les
                     exemples de la litt�rature en castillan ou en espagnol.}. Mon
                  �dition devra �tre accompagn�e d'une r�flexion pratique sur la
                  mani�re la plus efficace de mener le lecteur au point o� je veux aller:
                  il me faudra �viter l'�cueil d'une �dition complexe et peu
                  ergonomique, voire illisible pour le lecteur. Cette derni�re id�e
                  m'am�ne � parler du dernier point que je veux aborder dans ce projet de
                  th�se: les enjeux p�dagogiques, de diffusion et de communication de mon
                  travail. En effet, il me semble important de r�fl�chir � la
                  mani�re dont mon �dition va �tre diffus�e et lue. Synoptique,
                  elle aura besoin d'un minimum de guidage de ma part, au travers de rails qu'il me
                  faudra penser et construire (je donnerai un petit exemple, l'encadr�
                     ``options'' de mon �dition des prologues, qui indique quels
                  manuscrits sont proches selon des crit�res formels\footnote{\href{http://perso.ens-lyon.fr/matthias.gille-levenson/etude-prologues/Prologues.html\#option2}{http://perso.ens-lyon.fr/matthias.gille-levenson/etude-prologues/Prologues.html\#option2}.}). Ce point rejoint une id�e expos�e plus haut: il est
                  important de d�terminer de nouvelles fa�ons d'illustrer ou d'argumenter
                  un propos, quand on travaille sur des objets scientifiques nouveaux comme
                  l'�dition �lectronique. Comment faire le lien entre la th�se et
                  cette �dition? Ce question devra trouver une r�ponse au cours de mes
                  ann�es de doctorat. 
               
               
            
        Par ailleurs, je pense alimenter un carnet de recherche technique, qui
                  expliquerait au fur-et-�-mesure de l'avancement de mon travail les choix, les
                  enjeux qu'il implique. Il me permettra aussi de proposer des articles
                  d'explication des fondamentaux de tel ou tel outil informatique: je crois �
                  la formation autonome, mais je sais aussi par exp�rience que les premiers pas
                  avec un langage informatique, avec un standard, etc, sont les plus difficiles,
                  d'autant plus que les ressources en langue anglaise sont pr�dominantes. Je ne
                  s�pare pas ce versant p�dagogique et de communication/vulgarisation du
                  premier versant plus proprement scientifique: dans un projet comme le mien, la
                  clart� dans la documentation et l'explication de la d�marche du
                  chercheur est essentielle, pour des raisons de partage de connaissance et de
                  m�thode notamment.
            
         
         \section{IV - Localisation}
            
            
            
        Mon �tablissement de rattachement sera donc l'�NS de Lyon,
               �tablissement au sein duquel je souhaiterais pouvoir travailler avec Carlos
               Heusch, qui me dirige depuis mon M1. Professeur des Universit�s, chercheur au
               CIHAM [Centre Interuniversitaire d'Histoire et d'Arch�ologie
               M�di�vales] -- Histoire, arch�ologie, litt�ratures des mondes
               chr�tiens et musulmans m�di�vaux. Carlos Heusch a en effet beaucoup
               travaill� sur des questions de th�orisation politique (je pense par exemple
               � \href{http://calenda.org/268320}{Theorica II}, un
               atelier organis� par C. Heusch autour du th�me de la noblesse et du lignage
               en Castille), sur les concepts de chevalerie et de lignage en particulier, avec de
               nombreux articles consacr�s � ce sujet. L'oeuvre que je veux �tudier a
               aussi �t� comment�e � plusieurs reprises par Carlos
                     Heusch\footnote{ \cite{heusch_translation_2005},
                      \cite{heusch_caballeriayer_2010},  \cite{heusch_biografidebate:_2010}.}.
               Je pense aussi travailler avec Francisco Bautista, Professeur titulaire �
               l'Universit� de Salamanque, qui a �t� le co-directeur de mon
               M�moire de Master II et qui a beaucoup travaill� sur des questions
               id�ologiques et politiques au bas Moyen-�ge\footnote{Voir par exemple  \cite{bautista_narrativas_2014}.}. Sur
               l'autre versant important de ma th�se, le versant informatique, Marjorie
               Burghart, charg�e de recherche au CNRS, historienne m�di�viste et
               sp�cialiste en humanit�s num�riques, a accept� de me co-encadrer.
               Marjorie Burghart a �t� ma formatrice principale du point de vue de la TEI,
               lors des diff�rents stages que j'ai pu r�aliser � Lyon ou ailleurs
               dans le cadre du \href{https://www.digitalmanuscripts.eu/}{DEMM} (Digital Encoding of Medieval Manuscripts). En ce qui concerne le
               rattachement � un laboratoire de recherche, mon travail me semble s'inscrire
               assez bien dans le cadre de la recherche effectu�e au CIHAM: il se pr�te
               assez ais�ment � un rattachement � plusieurs axes de recherche, dont
               les trois suivants m'ont paru les plus significatifs pour mon travail: 
            \begin{itemize}\item Axe 2: \textsc{Pouvoir et autorit�}.\item  Axe 3: \textsc{Construction et communication des
                        savoirs}\footnote{Dont Carlos Heusch est co-responsable avec Laurence
                        Moulinier.}.\item Axe 4: \textsc{�criture, livre,
                     translation}.\item Axe transversal 5: \textsc{Humanit�s
                        num�riques}.\end{itemize}
            
            
        \noindent D'un point de vue plus p�dagogique, je souhaiterais
               pouvoir effectuer une activit� d'enseignement compl�mentaire dans mon
               �tablissement de rattachement, � savoir l'�NS de Lyon. 
            
            
        
         
      
      
         \section{V - Bibliographie indicative }
            
            
            
            
        \noindent Ci-dessous la bibliographie indicative de mon projet de
               th�se
                  \footnote{Une
                  bibliographie plus exhaustive est accessible au lien suivant: \href{http://perso.ens-lyon.fr/matthias.gille-levenson/these/biblio.html}{http://perso.ens-lyon.fr/matthias.gille-levenson/these/biblio.html}.}. ~ \printbibliography[heading=secbib]
            
         
      
   



\end{document}
  
  
  %% AVERTISSEMENT IMPORTANT: bien faire attention � l'arborescence des fichiers et � l'encodage homog�ne des caract�res (suppose de modifier les r�glages dans l'�diteur XML et dans le programme de gestion bibliographique en plus du programme de compilation LaTeX. 